\documentclass{article}
\usepackage[OT2, T2A, T1]{fontenc}
\usepackage[russian]{babel}
\usepackage{graphicx}
\usepackage{geometry}
\geometry{margin=2cm}

% PLEASE NOTICE THAT, ACCORDING TO YOUR TITLE LENGTH, YOU MAY WANT TO ADAPT THE VSPACEs, SO THAT THE FINAL PAGE WILL BE PLEASING YOUR TASTE!%

\title{TitlePage}
\author{Your Name}
\date{Put the date}

\begin{document}

\begin{titlepage}
    \begin{center}
            
        \Huge
        \textbf{КЫРГЫЗ-ТҮРК «МАНАС» УНИВЕРСИТЕТИ}
            
        \vspace{0.5cm}
        \LARGE
            ТАБИГЫЙ ИЛИМДЕР ФАКУЛЬТЕТИ
        
        %\vspace{0.5cm}
        \large
            %Колдонмо математика жана информатика бөлүмү
            
        \vspace{1cm}
         
        \includegraphics[width=0.5\textwidth]{Manas_logo.pdf}   
       
        
            
        \vspace{0.5cm}
            
        \Huge
            \textbf{ТЕЗ ЭСЕПТӨӨ ОЮНУ}
            
        \vspace{2cm}
            
        
     \end{center}  
     
\raggedright

\begin{tabular}{@{}l l@{}}
    \Large Студенттин аты-жөнү: & \Large Аскат Рахымбеков \\[1em]
    \Large Студенттик №: & \Large 2312.01002 \\[1em]
    \Large Бөлүмү: & \Large Колдонмо математика жана информатика \\
\end{tabular}

\Large
    %Thesis written by:
    %\vspace{0.125cm}\\
\Large
    %*Author name*
    
\end{titlepage}

\section{Проекттин максаты жана жалпы мүнөздөмөсү}
\begin{itemize}
    \item Тез эсептөө оюнун иштеп чыгуу.
    \item Бул проектте \textbf{«Math × Snake»} аттуу браузердик оюну ишке ашырат. Ал \textbf{JavaScript} (логика), \textbf{HTML5 Canvas} (графика) жана \textbf{CSS} (интерфейс) колдонуп жазылган. Оюндун идеясы — классикалык змейканын механикасын арифметикалык машыгуу менен айкалыштыруу: талаада кадимки «жемдин» ордуна бир нече \textbf{жооп варианттары} чыгат, ал эми үстүңкү панелде \textbf{математикалык туюнтма} көрсөтүлөт. Оюнчу змейканы туура \textbf{жоопту жегенге} багыттайт; туура эмес санды жесе — \textbf{жандары} кемийт.
\end{itemize}

Проект көңүл ачуу менен бирге \textbf{билим берүү} эффектин көздөйт: оюнчу оозеки эсепти, көңүл бурууну жана реакцияны оюндаштырылган форматта өнүктүрөт. Формат мектеп окуучуларына жана негизги арифметиканы кайталаган студенттерге ылайыктуу.

\section{Интерфейс жана колдонуучу тажрыйбасы (UX)}
Барак үч негизги зонадан турат:
\begin{enumerate}
  \item \textbf{Жогорку панель (Header / HUD):} Режим (\textit{Mode}), татаалдык (\textit{Level}), упай (\textit{Score}), жандар (\textit{Lives}) жана (эгер күйгүзүлсө) таймер көрсөтүлөт.
  \item \textbf{Негизги талаа (Canvas):} 20×20 тордон турган оюн талаасында жылаан, жооп плиткалары жана декоративдик эффекттер чийилет. Үстүнө оверлейлер чыгышы мүмкүн: \textit{Main menu}, \textit{How to play}, \textit{Level Complete}, \textit{Game Over}.
  \item \textbf{Башкаруу элементтери:} Меню, Пауза жана мобайл үчүн экрандык \textit{D-pad}.
\end{enumerate}

\section{Оюн логикасы}
Логиканын өзөгү \texttt{SnakeMathGame} классына топтолгон.

\subsection{Негизги параметрлер}
\begin{itemize}
  \item Тор өлчөмү: \textbf{20×20} клетка; клетка: \textbf{20 px}.
  \item Деңгээлди аяктоо үчүн туура жооптор: \textbf{10}.
  \item Баштапкы жандардын саны: \textbf{3}.
\end{itemize}

\subsection{Оюн режимдери}
\begin{itemize}
  \item \textbf{Addition} (кошуу), \textbf{Subtraction} (алып салуу), \textbf{Multiplication} (көбөйтүү), \textbf{Division} (калдыксыз бөлүү), \textbf{Mixed} (аралаш).
\end{itemize}

\subsection{Кыйынчылык деңгээлдери}
Деңгээл диапазондорго жана ылдамдыкка таасир этет:
\begin{itemize}
  \item \textbf{Easy} — бир орундуу; \textbf{Medium} — эки орундуу; \textbf{Mixed} — 1–99 аралаш; \textbf{Advanced} — үч орундууга чейин; \textbf{Expert} — тез оюн, ар бир туура жооптон кийин \textbf{ылдамдануу}.
\end{itemize}

\subsection{Мисалдарды түзүү жана жооп плиткалары}
\begin{itemize}
  \item Деңгээл башталганда туюнтма (\(a{+}b\), \(a{-}b\), \(a{\times}b\), \(a{\div}b\)) түзүлөт. Бөлүүдө жыйынтык дайыма \textbf{бүтүн} болушу үчүн \(a = b \cdot q\) принципи колдонулат.
  \item Талаада \textbf{4 позиция} тандалып, бирөөсүнө \textbf{туура жооп}, калгандарына \textbf{көңүлдү алаксытуучулар} коюлат.
  \item \textbf{Туура жооп} жегенде: упай +10, жылаандын узундугу +1; таймер күйүк болсо — \textbf{убакыт бонусу}; \textbf{Expert} деңгээлинде тик-интервал кыскарат (оюн тездейт).
  \item \textbf{Туура эмес} жооп — \textbf{жандардын саны -1}.
  \item \textbf{Sandbox mode} күйүк болсо, 10 туура жооп чектөөсү жок; болбосо 10го жеткенде \textit{деңгээл ийгиликтүү бүтүрүлдү}.
\end{itemize}

\subsection{Аяктоо шарттары}
\begin{itemize}
  \item Жандардын саны 0 болгондо — \textit{Game Over}.
  \item 10 туура жоопка жеткенде (Sandbox эмес) — \textit{Level Complete}.
\end{itemize}

\section{Башкаруу ыкмалары}
\begin{itemize}
  \item \textbf{Клавиатура:} жебелер же \textbf{WASD}; тескери бурулуп куйрукка урунбоо үчүн текшерүү бар.
  \item \textbf{Экрандык D-pad:} мобайл үчүн чоң баскычтар.
  \item \textbf{Сенсордук свайп}.
\end{itemize}

\section{Техникалык ишке ашыруу}
\subsection{Графика жана рендер}
\begin{itemize}
  \item Чийүү \textbf{Canvas 2D API} аркылуу: тор, змейка, плиткалар, чек кырлары.
  \item Плиткалар: тегиз фон, ичке чек, ортодо сан; туура жоопко (эгер күйүк) \textbf{жашыл} оттенок.
  \item Змейканын башы өзгөчө түстө, «көздөрү» бар.
\end{itemize}

\subsection{Абал (state) жана цикл}
\begin{itemize}
  \item Абал \texttt{gameState} ичинде: позициялар, багыт, өсүү кезеги, таймерлер, упай/жашоо, учурдагы туюнтма/жооп, плиткалар, ылдамдык ж.б.
  \item Негизги цикл — \texttt{setInterval} менен \texttt{tick()}; анда кыймыл, кагылышуу (дуң/өзү менен), плитка жеш, HUD жаңыртуу жүрөт.
  \item Пауза/Resume логикасы оверлейлерге байланган; таймер секунданын ичинде кемийт, 0 болсо — \textbf{жашоо -1}.
\end{itemize}

\subsection{Параметрлер жана туруктуулук}
\begin{itemize}
  \item Колдонуучу тандоолору (\textit{mode}, \textit{level}, \textit{timerEnabled}) \texttt{localStorage} аркылуу сакталат/жүктөлөт.
  \item Ылдамдык деңгээлге жараша \texttt{tickInterval} менен коюлат; \textbf{Expert} режиминде туура жооптон кийин динамикалык кыскартылат.
\end{itemize}

\section{Билим берүү багыты}
Бул прототип классикалык механиканы \textbf{эсеп чыгаруучу машыгуу} менен бириктирет:
\begin{itemize}
  \item \textbf{Арифметика:} тез эсептөөнү машыктырат;
  \item \textbf{Көп тапшырма:} туюнтманы чечип, ошол эле учурда навигация кылуу — көңүл топтоону/реакцияны бекемдейт;
  \item \textbf{Оюндаштыруу:} мотивацияны жана үзгүлтүксүз машыгууну жогорулатат.
\end{itemize}
Колдонуу чөйрөлөрү: сабактарда, машыгуу сабатында, онлайн-курстарда кошумча курал.

\section{Күчтүү жактар жана жакшыртуу сунуштары}
\subsection*{Күчтүү жактар}
\begin{itemize}
  \item Толук \textbf{клиент-тарап} чечим; жайгаштыруу оңой.
  \item ПК/мобайлда ыңгайлуу: клавиатура, \textit{D-pad}, свайп.
  \item Ийкемдүү параметрлер: режимдер, деңгээлдер, таймер, подсказка, Sandbox.
\end{itemize}

\subsection*{Жакшыртуу идеялары}
\begin{enumerate}
  \item \textbf{Анимация цикли:} \texttt{setInterval} ордуна \texttt{requestAnimationFrame} — кадрлардын туруктуулугу/энергия үнөмдүүлүгү.
  \item \textbf{Жооп генерациясы:} дистракторлорду семантикалык жакын кылуу (көбөйтүүдө факторлор, бөлүүдө бөлгүчтөр).
  \item \textbf{Жогорку темалар:} пайыздар, даражалар, модуль, аралаш сандар (жөнөкөйдөштүрүлгөн режимде).
  \item \textbf{Прогресс/лидерборд:} жергиликтүү же булут аркылуу (мис., Firebase).
  \item \textbf{Үндөр/вибро:} туура/ката жоопко сигнал, мобайлда виброфидбэк.
  \item \textbf{Адаптивдүү татаалдаштыруу:} тактыкка жараша диапазондорду өзгөртүү.
  \item \textbf{Код структурасы:} модулга бөлүү (рендер, инпут, логика, генерация) — тесттөөгө ыңгайлуу.
\end{enumerate}

\section{Жыйынтык}
\textbf{«Snake × Math»} — Canvas API жана заманбап CSS менен курулган, толук иштей турган \textbf{edutainment}-оюн. Ал классикалык механиканы туура колдонуп, билим берүү элементин табигый кошот: оюнчу жоопту тез эсептеп, туура санды тандоого үйрөнөт.

Практикалык колдонуу сценарийлери:
\begin{itemize}
  \item Мектеп/кружок үчүн математика боюнча \textbf{кыска машыгуу} куралы;
  \item Информатика/веб-разработка сабактарында \textbf{демо-проект};
  \item Онлайн-курстарда \textbf{мини-оюн} катары мотивация куралы.
\end{itemize}
Проект оңой кеңейтилет: жаңы мисал түрлөрү, лидерборд, адаптивдүү кыйынчылык, мультимедиялык эффекттер.

\end{document}